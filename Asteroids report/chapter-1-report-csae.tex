\chapter{\label{ch:ch01}ГЛАВА 1. Обзор Godot} % Нужно сделать главу в содержании заглавными буквами

\section{\label{sec:ch01/sec01}Раздел 1. Возможности Godot}

\subsection{\label{subsec:ch01/sec01/sub01}Описание движка}

Godot - это универсальный и простой в освоении игровой движок, который позволяет создавать как 2D, так и 3D игры. Он обладает широким спектром возможностей, делающих его привлекательным выбором как для начинающих, так и для опытных разработчиков.

\subsection{\label{subsec:ch01/sec01/sub02}2D и 3D возможности Godot}

\textbf{2D:}

\begin{itemize}
    \item Создание 2D персонажей, окружения и интерфейсов.
    \item Плавная анимация с использованием спрайтов, скелетной анимации и tween-инструментов.
    \item Физика 2D с реалистичными столкновениями и взаимодействиями.
    \item Tilemaps для создания уровней с повторяющимися элементами.
    \item Камеры с эффектами параллакса, приближения и тряски.
    \item Частицы для создания визуальных эффектов, таких как взрывы, дождь и огонь.
\end{itemize}

\textbf{3D:}

\begin{itemize}
    \item 3D моделирование и импорт из популярных форматов (OBJ, FBX и т.д.).
    \item PBR-шейдеры для реалистичного освещения и материалов.
    \item Физика 3D с Bullet Physics Engine.
    \item Анимация персонажей с использованием скелетной анимации.
    \item Редактор для создания 3D ландшафтов.
    \item Эффекты пост-обработки для лучшего визуального качества игры.
\end{itemize}


\subsection{\label{subsec:ch01/sec01/sub03}Система, основанная на "нодах"}

Godot использует уникальную node-based систему для организации игровых объектов и их взаимодействия. Это позволяет разработчикам легко создавать сложные сцены и управлять объектами через иерархию узлов. У каждого "нода" есть свои параметры и "дети" которые наследуют некоторые характеристики "родителя", что позволяет создавать различные взаимодействия между объектами чуть легче, чем в других движках.

\subsection{\label{subsec:ch01/sec01/sub04}Языки программирование в Godot}

GDScript, специально разработанный для использования в Godot, обладает простым и интуитивно понятным синтаксисом (схожим с Python), что делает его идеальным выбором для новичков и тех, кто только начинает свой путь в разработке игр. Благодаря тесной интеграции с движком, GDScript предоставляет простой доступ ко всем функциям и возможностям Godot, что ускоряет процесс разработки и позволяет сосредоточиться на творческой части проекта, минуя множество нюансов, связанных с управлением ресурсами и событиями. 

Также Godot поддерживает C\# мощный и гибкий язык программирования, широко используемый в индустрии разработки игр. Разработчики, уже имеющие опыт работы с C\# или желающие использовать существующий код и ресурсы, могут воспользоваться этим языком для создания игр в Godot, что сделает процесс адаптации и разработки более комфортным и эффективным.

Также в Godot можно использовать и другие языки с помощью официального аддона GDExtension, он поддерживает такие языки, как Rust, Go, Haxe, D, Swift, C++. 

