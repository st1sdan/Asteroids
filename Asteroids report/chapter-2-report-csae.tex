\chapter{\label{ch:ch02}ГЛАВА 2. Разработка игры}

\section{\label{sec:ch02/sec01}Раздел 1. Проектирование игрового процесса и механик}

\subsection{\label{subsec:ch02/sec01/sub01}Подраздел 1. Анализ классической Asteroids и определение основных игровых механик}

Анализ основной игры "Asteroids". Ключевые механики:
\begin{itemize}
    \item Управление игровым кораблем в открытом космосе с возможностью перемещения во всех направлениях.
    \item Разрушение астероидов при помощи снарядов, ведущее к их разделению на более мелкие части.
    \item Опасность столкновения с астероидами и другими объектами на игровом поле, что приводит к потере жизней игрока или поражению.
    \item Система отсчета очков, мотивирующая игрока к достижению новых рекордов и улучшению своих результатов.
\end{itemize}

Новые механики:
\begin{itemize}
    \item Выбор сложности в меню настроек, от сложности зависит количество осколков на которые разделяется астероид, их размер, и количество очков, которое дается за их уничтожение.
    \item Онлайн таблица лидеров в реальном времени, после конца игры можно записать себя в таблицу лидеров под любым именем.
    \item Система бонусов и усилений, которые помогают игроку в сражении с астероидами и увеличивают его шансы на выживание.
    \item Различные визуальные и звуковые эффекты для лучшего игрового опыта.
\end{itemize}


\subsection{\label{subsec:ch02/sec01/sub02}Подраздел 2. Определение целей игры, основных действий игрока и правил игрового процесса}

Основные цели, задачи и правила игры:
\begin{itemize}
    \item Цель игры: выжить как можно дольше, уничтожая астероиды и избегая столкновений с ними и другими объектами.
    \item Основные действия игрока: управление кораблем, стрельба по астероидам, уклонение от опасностей и сбор бонусов.
    \item Правила игрового процесса: игрок начинает игру в центре экрана и пытается набрать максимальное количество очков, избегая столкновений и уничтожая астероиды.
\end{itemize}

\subsection{\label{subsec:ch02/sec01/sub03}Подраздел 3. Создание концепции игры, включая уровни сложности, прогрессию и целевые достижения}

Общая концепция игры "Asteroids", включающая:
\begin{itemize}
    \item Уровни сложности: планируется создание нескольких уровней, каждый из которых будет характеризоваться увеличивающимся количеством и скоростью астероидов, а также другими усложнениями игрового процесса.
    \item Прогрессия: игра будет предусматривать систему постепенного увеличения сложности, чтобы поддерживать интерес игрока и давать ему возможность постоянно улучшать свои навыки.
    \item Целевые достижения: будут определены определенные цели и достижения, например, достижение определенного количества очков, продержаться определенное время или уничтожить определенное количество астероидов.
\end{itemize}

\section{\label{sec:ch02/sec01}Раздел 2. Проектирование и реализация игрового корабля}

- Разработка внешнего вида игрового корабля с учетом эстетических и функциональных требований
- Создание анимаций для игрового корабля, включая анимацию движения, выстрелов и уничтожения
- Реализация управления игровым кораблем с использованием клавиатуры, мыши или других устройств ввода
- Программирование взаимодействия игрового корабля с окружающим миром, такое как столкновения с астероидами и другими игровыми объектами

\subsection{\label{subsec:ch02/sec01/sub01}Подраздел 1. Разработка внешнего вида игрового корабля}

Корабль представлен простым белым треугольником, сзади которого при движении выпускаются эффекты огня (работы двигателя), стреляет корабль из своего "носа" желтыми круглыми пулями.

\begin{figure}
    \centering
    \includegraphics[width=0.5\linewidth]{images/spaceship.png}
    \caption{Космический корабль}
    \label{fig:enter-label}
\end{figure}

\subsection{\label{subsec:ch02/sec01/sub02}Подраздел 2. Реализация управления игровым кораблем}


Реализация управления игровым кораблем в скрипте player.gd~\ref{code:example01}.
\begin{code}
\captionof{listing}{\centering\label{code:example01}Код для движения корабля}
\vspace{-\baselineskip}\begin{minted}{C}
func move(delta):
    # Задаем направление поворота персонажа
    rotation_direction = Input.get_axis("ui_left", "ui_right")
    # Проверяем нажата ли клавиша для движения вперед
    is_moving = Input.is_action_pressed("forward")

    # Поворот персонажа при нажатии клавиш
    if rotation_direction > 0:
        rotation += rotation_speed * delta
    elif rotation_direction < 0:
        rotation -= rotation_speed * delta

    # Если клавиша вперед нажата двигаемся вперед относительно поворота персонажа
    # И замедляемся, при отжатии клавиши
    if is_moving:
        velocity += Vector2(0, 1).rotated(rotation) * speed
    else:
        velocity = velocity.move_toward(Vector2.ZERO, deceleration)
    velocity = velocity.clamp(-max_speed_vector, max_speed_vector)
\end{minted}
\end{code}

Функция "move" находится в основной функции \_physics\_process(delta), которая обновляется каждый кадр, где delta - разность во времени между двумя кадрами.

\subsection{\label{subsec:ch02/sec01/sub03}Подраздел 3. Программирование взаимодействия игрового корабля с окружающим миром}

При столкновении с астероидом конец игры и переход в следующую сцену~\ref{code:example02}.
\begin{code}
\captionof{listing}{\centering\label{code:example02}Код для конца игры}
\vspace{-\baselineskip}\begin{minted}{C}
func _on_collision_player_body_entered(body):
    if body is Enemy:
        die()


func die():
    get_tree().change_scene_to_file("res://Scenes/end_screen.tscn")
\end{minted}
\end{code}


Перемещение персонажа при достижении края экрана~\ref{code:example03}.
\begin{code}
\captionof{listing}{\centering\label{code:example03}Код для перемещения персонажа при достижении края экрана}
\vspace{-\baselineskip}\begin{minted}{C}
func out_of_bounds():
    position = position.posmodv(screen_size)
\end{minted}
\end{code}


Реализация стрельбы при нажатии определенной кнопки и реализация улучшения "тройной выстрел"~\ref{code:example03}.
\begin{code}
\captionof{listing}{\centering\label{code:example03}Код для стрельбы}
\vspace{-\baselineskip}\begin{minted}{C}
func shoot():
    bullet_timer -= 1

    if !can_shoot and bullet_timer <= 0:
        can_shoot = true
        bullet_timer = 20

    if can_shoot and Input.is_action_just_pressed("shoot"):
        $AudioStreamPlayer.pitch_scale = randf_range(0.8, 1.6)
        $AudioStreamPlayer.playing = true

        can_shoot = false
        var bullet = BULLET_PATH.instantiate()
        if Globals.triple_shot:
            var bullet2 = bullet.duplicate()
            bullet2.global_position = $Marker2D.global_position + Vector2(10, 10)
            bullet2.velocity = Vector2(0, 1).rotated(rotation) * bullet_speed
            bullet2.scale = Vector2(0.1, 0.1)
            var bullet3 = bullet.duplicate()
            bullet3.global_position = $Marker2D.global_position + Vector2(-10, -10)
            bullet3.velocity = Vector2(0, 1).rotated(rotation) * bullet_speed
            bullet3.scale = Vector2(0.1, 0.1)
            get_parent().add_child(bullet2)
            get_parent().add_child(bullet3)
        bullet.scale = Vector2(0.1, 0.1)

        get_parent().add_child(bullet)
        bullet.position = $Marker2D.global_position
        bullet.velocity = Vector2(0, 1).rotated(rotation) * bullet_speed
\end{minted}
\end{code}

\section{\label{sec:ch02/sec03}Раздел 3. Добавление звуковых эффектов и музыки}

\begin{itemize}
    \item Подбор и добавление звуковых эффектов для различных игровых событий, таких как выстрелы, взрывы и столкновения с помощью сайта \cite{https://sfxr.me/}
    \item Интеграция фоновой музыки для создания атмосферы игры и поддержания интереса игрока. Фоновая музыка (с разрешения автора) \cite{https://www.youtube.com/watch?v=-35AC-FPoAA}
    \item Настройка звуковых уведомлений и звуковых сигналов для обратной связи с пользователем, также с помощью сайта \cite{https://sfxr.me/}
\end{itemize}
 